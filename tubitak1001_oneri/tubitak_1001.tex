\documentclass[a4paper]{article}

\usepackage{tikz}
\usepackage{pgfplots}
\usetikzlibrary{intersections, pgfplots.fillbetween}
\usetikzlibrary{shapes.multipart}
\usetikzlibrary{decorations.pathreplacing,decorations.markings,calc,positioning,backgrounds}
\usetikzlibrary{intersections,fadings}
\usepgfplotslibrary{fillbetween}
\usetikzlibrary{arrows,shapes.arrows,arrows.meta}
\usepackage{lipsum}
%%%%%%%%%%%%%%%%%%%%%%%%%%%%%%%%%%%%%%%%%
\definecolor{Gray}{gray}{0.85}
%%%%%%%%%%%%%%%%%%%%%%%%%%%%%%%%%%%%%%%%%
\newcommand{\dif}{\mathrm{d}}
%%%%%%%%%%%%%%%%%%%%%%%%%%%%%%%%%%%%%%%%
\fancypagestyle{mylandscape}{
\fancyhf{} %Clears the header/footer
\fancyhead[C]{
\makebox[\textwidth][l]{
\rlap{
\smash{
\raisebox{-4.87in}{
\rotatebox{90}{\includegraphics{tubitak.jpg}}
}}}
}}
\fancyfoot[L]{% Footer
\smash{
\hspace{17.5cm}
\rotatebox{90}{\small 3501BF-01 Güncelleme Tarihi: 30/12/2023}}}
\fancyfoot[C]{
\makebox[\textwidth][r]{% Right
%  \rlap{
%  \hspace{.75cm}% Push out of margin by \footskip
%  \smash{% Remove vertical height
%  \raisebox{4.87in}{% Raise vertically
%  \rotatebox{90}{\small 3501BF-01 Güncelleme Tarihi: 30/09/2019}}}}
%
   \rlap{
   \hspace{.75cm}% Push out of margin by \footskip
    \smash{% Remove vertical height
    \raisebox{4.87in}{% Raise vertically
    \rotatebox{90}{\thepage}}}}
    }}% Rotate counter-clockwise
\renewcommand{\headrulewidth}{0pt}% No header rule
\renewcommand{\footrulewidth}{0pt}% No footer rule
}
%%%%%%%%% JOURNALS %%%%%%%%%%%%%%%%%%%%%%
\def\apj{Astrophysical Journal}
\def\apjs{Astrophysical Journal Supplements}
\def\apjl{Astrophysical Journal Letters}
\def\apss{Astrophysics and Space Science}
\def\jcap{Journal of Cosmology and Astroparticle Physics}
\def\prd{Physical Review D}
\def\prc{Physical Review C}
\def\nat{Nature}
\def\aj{The Astronomical Journal}
\def\aap{Astronomy and Astrophysics}
\def\mnras{Monthly Notices of Royal Astronomical Society}
\def\plb{Physics Letters B}
\def\pla{Physics Letters A}
\def\physrep{Physics Reports}



%\def\aj           Astronomical Journal
\def\actaa{Acta Astronomica}
\def\araa{Annual Review of Astron and Astrophys}
%\def\apj          Astrophysical Journal
%\def\apjl         Astrophysical Journal, Letters
%\def\apjs         Astrophysical Journal, Supplement
%\def\ao           Applied Optics
%\def\apss         Astrophysics and Space Science
%\def\aap          Astronomy and Astrophysics
%\def\aapr         Astronomy and Astrophysics Reviews
%\def\aaps         Astronomy and Astrophysics, Supplement
\def\azh{Astronomicheskii Zhurnal}
%\def\baas         Bulletin of the AAS
%\def\caa          Chinese Astronomy and Astrophysics
%\def\cjaa         Chinese Journal of Astronomy and Astrophysics
%\def\icarus       Icarus
%\def\jcap         Journal of Cosmology and Astroparticle Physics
%\def\jrasc        Journal of the RAS of Canada
%\def\memras       Memoirs of the RAS
%\def\mnras        Monthly Notices of the RAS
\def\na{New Astronomy}
\def\nar{New Astronomy Review}
%\def\pra          Physical Review A: General Physics
%\def\prb          Physical Review B: Solid State
%\def\prc          Physical Review C
%\def\prd          Physical Review D
%\def\pre          Physical Review E
\def\prl{Physical Review Letters}
%\def\pasa         Publications of the Astron. Soc. of Australia
\def\pasp{Publications of the ASP}
\def\pasj{Publications of the ASJ}
%\def\rmxaa        Revista Mexicana de Astronomia y Astrofisica
%\def\qjras        Quarterly Journal of the RAS
%\def\skytel       Sky and Telescope
%\def\solphys      Solar Physics
\def\sovast{Soviet Astronomy}
\def\ssr{Space Science Reviews}
%\def\zap          Zeitschrift fuer Astrophysik
%\def\nat          Nature
%\def\iaucirc      IAU Cirulars
\def\aplett{Astrophysics Letters}
%\def\apspr        Astrophysics Space Physics Research
%\def\bain         Bulletin Astronomical Institute of the Netherlands
%\def\fcp          Fundamental Cosmic Physics
%\def\gca          Geochimica Cosmochimica Acta
%\def\grl          Geophysics Research Letters
%\def\jcp          Journal of Chemical Physics
\def\jgr{Journal of Geophysics Research}
%\def\jqsrt        Journal of Quantitiative Spectroscopy and Radiative Transfer
%\def\memsai       Mem. Societa Astronomica Italiana
%\def\nphysa       Nuclear Physics A
%\def\physrep      Physics Reports
%\def\physscr      Physica Scripta
%\def\planss       Planetary Space Science
%\def\procspie     Proceedings of the SPIE
%%%%%%%%%%%%%%%%%%%%%%%%%%%%%%%%%%%%%%%%%%
\usepackage{tcolorbox}
\tcbuselibrary{skins,breakable}

\newtcolorbox{ozet}[1][]{%
colback=white,colframe=black,
width=\textwidth,
sharp corners,
breakable,
bottomrule=0.5pt, 
toprule=0.5pt,  % thicknesses of various lines
leftrule=0.5pt, 
rightrule=0.5pt,
left=6pt,
right=6pt,  
segmentation style={solid},  
before upper={\parindent15pt},
#1
}

\newtcolorbox{summary}[2][]{%
colbacktitle=white,
coltitle=black,
colback=white,
colframe=black,
width=\textwidth,
sharp corners,
breakable,
titlerule=0.5pt,
bottomrule=0.5pt, 
toprule=0.5pt,  % thicknesses of various lines
leftrule=0.5pt, 
rightrule=0.5pt,
left=6pt,
right=6pt,  
segmentation style={solid},  
before upper={\parindent15pt},
title={\textbf{Title:} #2}
#1
}
%%%%%%%%%%%%%%%%%%%%%%%%%%%%%%%
\usepackage{pgfmath}  
\usepackage{xstring}  
\usepackage{refcount} 

\newcommand{\refte}[1]{%
 \def\eqnum{\getrefnumber{#1}}%
 \ifx\getrefnumber{#1}\relax%
   \ref{#1}'te%
  \else%
  \pgfmathparse{int(mod(\eqnum , 10))}% Extract last digit
  \let\lastdigit\pgfmathresult%
  \pgfmathparse{int(mod(\eqnum , 100))} % Extract last digit
  \let\lasttwodigit\pgfmathresult%
  \ref{#1}'%
  \ifnum\lastdigit=0%
      \ifnum\lasttwodigit=0 de%
      \else\ifnum\lasttwodigit=10 da%
      \else\ifnum\lasttwodigit=20 de%
      \else\ifnum\lasttwodigit=30 da%
      \else\ifnum\lasttwodigit=40 ta%
      \else\ifnum\lasttwodigit=50 de%
      \else\ifnum\lasttwodigit=60 ta%
      \else\ifnum\lasttwodigit=70 te%
      \else\ifnum\lasttwodigit=80 de%
      \else\ifnum\lasttwodigit=90 da%
      \fi\fi\fi\fi\fi\fi\fi\fi\fi\fi
  \else\ifnum\lastdigit=1 de%
  \else\ifnum\lastdigit=2 de%
  \else\ifnum\lastdigit=3 te%
  \else\ifnum\lastdigit=4 te%
  \else\ifnum\lastdigit=5 te%
  \else\ifnum\lastdigit=6 da%
  \else\ifnum\lastdigit=7 de%
  \else\ifnum\lastdigit=8 de%
  \else\ifnum\lastdigit=9 da%
  \fi\fi\fi\fi\fi\fi\fi\fi\fi\fi
\fi
}

\newcommand{\refin}[1]{%
 \def\eqnum{\getrefnumber{#1}}%
 \ifx\getrefnumber{#1}\relax%
   \ref{#1}'in%
  \else%
  \pgfmathparse{int(mod(\eqnum , 10))}% % Extract last digit
  \let\lastdigit\pgfmathresult%
  \ref{#1}'%
  \ifnum\lastdigit=0%
      \pgfmathparse{int(mod(\eqnum , 100))}% % Extract last digit
      \let\lasttwodigits\pgfmathresult%
      \ifnum\lasttwodigits=0 ün%
      \else\ifnum\lasttwodigits=10 un% 
      \else\ifnum\lasttwodigits=20 nin%
      \else\ifnum\lasttwodigits=30 un%
      \else\ifnum\lasttwodigits=40 ın%
      \else\ifnum\lasttwodigits=50 nin%
      \else\ifnum\lasttwodigits=60 ın%
      \else\ifnum\lasttwodigits=70 in%
      \else\ifnum\lasttwodigits=80 nin%
      \else\ifnum\lasttwodigits=90 nın%
      \fi\fi\fi\fi\fi\fi\fi\fi\fi\fi 
  \else\ifnum\lastdigit=1 in%
  \else\ifnum\lastdigit=2 nin%
  \else\ifnum\lastdigit=3 ün%
  \else\ifnum\lastdigit=4 ün%
  \else\ifnum\lastdigit=5 in%
  \else\ifnum\lastdigit=6 nın%
  \else\ifnum\lastdigit=7 nin%
  \else\ifnum\lastdigit=8 in%
  \else\ifnum\lastdigit=9 un%
  \fi\fi\fi\fi\fi\fi\fi\fi\fi\fi
\fi
}


\newcommand{\refi}[1]{%
 \def\eqnum{\getrefnumber{#1}}%
 \ifx\getrefnumber{#1}\relax%
   \ref{#1}'i%
  \else%
  \pgfmathparse{int(mod(\eqnum , 10))}% % Extract last digit
  \let\lastdigit\pgfmathresult%
  \ref{#1}'%
  \ifnum\lastdigit=0%
      \pgfmathparse{int(mod(\eqnum , 100))}% % Extract last digit
      \let\lasttwodigits\pgfmathresult%
      \ifnum\lasttwodigits=0 ü%
      \else\ifnum\lasttwodigits=10 u% 
      \else\ifnum\lasttwodigits=20 yi%
      \else\ifnum\lasttwodigits=30 u%
      \else\ifnum\lasttwodigits=40 ı%
      \else\ifnum\lasttwodigits=50 yi%
      \else\ifnum\lasttwodigits=60 ı%
      \else\ifnum\lasttwodigits=70 i%
      \else\ifnum\lasttwodigits=80 i%
      \else\ifnum\lasttwodigits=90 ı%
      \fi\fi\fi\fi\fi\fi\fi\fi\fi\fi % 
  \else\ifnum\lastdigit=1 i%
  \else\ifnum\lastdigit=2 yi%
  \else\ifnum\lastdigit=3 ü%
  \else\ifnum\lastdigit=4 ü%
  \else\ifnum\lastdigit=5 i%
  \else\ifnum\lastdigit=6 yı%
  \else\ifnum\lastdigit=7 yi%
  \else\ifnum\lastdigit=8 i%
  \else\ifnum\lastdigit=9 u%
  \fi\fi\fi\fi\fi\fi\fi\fi\fi\fi
\fi
}


\newcommand{\refe}[1]{%
 \def\eqnum{\getrefnumber{#1}}%
 \ifx\getrefnumber{#1}\relax%
   \ref{#1}'e%
  \else%
  \pgfmathparse{int(mod(\eqnum , 10))}% % Extract last digit
  \let\lastdigit\pgfmathresult%
  \ref{#1}'%
  \ifnum\lastdigit=0%
      \pgfmathparse{int(mod(\eqnum , 100))}% % Extract last digit
      \let\lasttwodigits\pgfmathresult%
      \ifnum\lasttwodigits=0 e%
      \else\ifnum\lasttwodigits=10 a% 
      \else\ifnum\lasttwodigits=20 ye%
      \else\ifnum\lasttwodigits=30 a%
      \else\ifnum\lasttwodigits=40 a%
      \else\ifnum\lasttwodigits=50 ye%
      \else\ifnum\lasttwodigits=60 a%
      \else\ifnum\lasttwodigits=70 e%
      \else\ifnum\lasttwodigits=80 e%
      \else\ifnum\lasttwodigits=90 a%
      \fi\fi\fi\fi\fi\fi\fi\fi\fi\fi
  \else\ifnum\lastdigit=1 e%
  \else\ifnum\lastdigit=2 ye%
  \else\ifnum\lastdigit=3 e%
  \else\ifnum\lastdigit=4 e%
  \else\ifnum\lastdigit=5 e%
  \else\ifnum\lastdigit=6 ya%
  \else\ifnum\lastdigit=7 ye%
  \else\ifnum\lastdigit=8 e%
  \else\ifnum\lastdigit=9 a%
  \fi\fi\fi\fi\fi\fi\fi\fi\fi\fi
  \fi
}

\newcommand{\reften}[1]{%
 \def\eqnum{\getrefnumber{#1}}%
 \ifx\getrefnumber{#1}\relax%
   \ref{#1}'ten%
  \else%
  \pgfmathparse{int(mod(\eqnum , 10))}% % Extract last digit
  \let\lastdigit\pgfmathresult%
  \ref{#1}'%
  \ifnum\lastdigit=0% 
      \pgfmathparse{int(mod(\eqnum , 100))}% % Extract last digit
      \let\lasttwodigits\pgfmathresult%
      \ifnum\lasttwodigits=0 dan%
      \else\ifnum\lasttwodigits=10 dan% 
      \else\ifnum\lasttwodigits=20 den%
      \else\ifnum\lasttwodigits=30 dan%
      \else\ifnum\lasttwodigits=40 tan%
      \else\ifnum\lasttwodigits=50 den%
      \else\ifnum\lasttwodigits=60 tan%
      \else\ifnum\lasttwodigits=70 ten%
      \else\ifnum\lasttwodigits=80 den%%
      \else\ifnum\lasttwodigits=90 dan%
      \fi\fi\fi\fi\fi\fi\fi\fi\fi\fi%
  \else\ifnum\lastdigit=1 den%
  \else\ifnum\lastdigit=2 den%
  \else\ifnum\lastdigit=3 ten%
  \else\ifnum\lastdigit=4 ten%
  \else\ifnum\lastdigit=5 ten%
  \else\ifnum\lastdigit=6 dan%
  \else\ifnum\lastdigit=7 den%
  \else\ifnum\lastdigit=8 den%
  \else\ifnum\lastdigit=9 dan%
  \fi\fi\fi\fi\fi\fi\fi\fi\fi\fi%
\fi%
}

\input{defs_optional}

\newcommand{\forceindent}{\leavevmode{\parindent=1em\indent}}
\hyphenation{fortran et-ki-le-şim-le-ri-ni yıl-dız yıl-dız-la-rı yıl-dız-la-rı-nı}

\pagenumbering{arabic}

\begin{document}
\begin{center}
\fcolorbox{white}{lightgray}{
\parbox{0.98\textwidth}{
\centering{\Large\textbf{
1001 - BİLİMSEL VE TEKNOLOJİK ARAŞTIRMA PROJELERİNİ DESTEKLEME PROGRAMI
PROJE BAŞVURU FORMU
}}
}
}

\end{center}
\vspace{0.3cm}

\begin{center}
\begin{tabular}{|m{0.98\textwidth}|}
\hline\vspace{1pt}
\textbf{Proje Başlığı:}  \\
\hline\vspace{1pt}
\textbf{Proje Yürütücüsü:}  \\
\hline\vspace{1pt}
\textbf{Projenin Yürütüleceği Kurum/Kuruluş:}  \\
\hline
\end{tabular}
\end{center}

%Başvuru formunun Arial 9 yazı tipinde, her bir konu başlığı altında verilen açıklamalar göz önünde bulundurularak hazırlanması ve ekler hariç toplam 25 sayfayı geçmemesi gerekmektedir. Dosya depolama/paylaşım sistemlerindeki dosyalara ve/veya web sayfalarına link verilerek proje içeriğinin başvuru formu sınırları dışında ayrı bir alanda paylaşılması halinde, proje bilimsel değerlendirmeye alınmadan iade edilir. Form değişiklikleri izle modunda bırakılmamalı ve yorum içermemelidir. Formun içeriği ayrı bir ek olarak farklı dosyada paylaşılmamalıdır. Proje önerisine ilişkin tüm bilgilerin formda yer alan ilgili bölüme eklenmesi ve formun nihai halinin tek bir dosya olarak başvuru sistemine yüklenmesi gerekmektedir. Proje önerisi değerlendirme formuna ulaşmak için tıklayınız.

%%%%%%%%%%%%%%%%%%%%%%%%%%%%%%%%%%
\vspace{0.3cm}
\noindent\textbf{ÖZET}
%\vspace{0.3cm}

%Türkçe ve İngilizce özetlerin projenin (a) özgün değeri, (b) yöntemi, (c) yönetimi ve (d) yaygın etkisinin ana hatlarını kapsaması beklenir. Her bir özet 600 kelime ile sınırlandırılmalıdır. Bu bölümün en son yazılması önerilir.

\begin{ozet}
\begin{center}
\textbf{Proje Özeti}
\end{center}

\lipsum[1-3]

\tcbline*
\noindent\textbf{Anahtar Kelimeler:} 
kelime1, ... 
\end{ozet}
\vspace{0.3cm}

\begin{summary}{\textbf{Title}}
\begin{center}
\textbf{Summary}
\end{center}

\lipsum[1-4]
\tcbline*
\noindent\textbf{Keywords:} 
keyword1,... 
\end{summary}


%%%%%%%%%%%%%%%%%%%%%%%%%%%%%%%%%%%%%%%%%%%%%%%%%%%%%%%%%%%%
\section{ÖZGÜN DEĞER}

\subsection{Konunun Önemi ve Projenin Özgün Değeri:}
%Proje önerisinde ele alınan konunun kapsamı ve önemi nitel ve/veya nicel verilerle desteklenerek literatürün eleştirel bir değerlendirmesi ile açıklanır. Projenin literatürdeki hangi eksikliği nasıl gidereceği veya hangi soruna nasıl bir çözüm geliştireceği, literatürdeki çalışmalardan farklı olarak ilgili bilim veya teknoloji alan(lar)ına kavramsal, kuramsal ve/veya metodolojik olarak ne gibi özgün katkılarda bulunacağı açıklanır. Kaynaklar https://tubitak.gov.tr/tr/duyuru/bibliyografik-verilerin-duzenlenmesi sayfasındaki açıklamalara uygun olarak EK-1’de verilir.

\begin{framed}
\lipsum[1-5]
\end{framed}


\subsection{Araştırma Sorusu ve/veya Hipotezi:}
%Projenin ele aldığı problem(ler), araştırma sorusu ve/veya hipotezi açık bir şekilde ortaya konulur.

\begin{framed}
\lipsum[1-5]
\end{framed}

\subsection{Amaç ve Hedefler:}
%Projenin amacı ve hedefleri açık, ölçülebilir, gerçekçi ve proje süresince ulaşılabilir nitelikte olacak şekilde açıklanır.

\begin{framed}
\lipsum[1-5]
\end{framed}
%%%%%%%%%%%%%%%%%%%%%%%%%%%%%%%%%%%%%%%%%%%%%%%%%%%%%%%%%%%%%%%%%%%%%%%%%%%%%%%%%%%%%%%

%%%%%%%%%%%%%%%%%%%%%%%%%%%%%%%%%%%%%%%%%%%%%%%%%%%%%%%%%%%%%%%%%%%%%%%%%%%%%%%%%%%%%%%
\section{YÖNTEM}
%Projede uygulanacak yöntem ve araştırma teknikleri (veri toplama araçları ve analiz yöntemleri dâhil) ilgili literatüre atıf yapılarak tercih sebepleri ile birlikte ayrıntılı bir şekilde açıklanır. Yöntemin projenin amaç ve hedeflerine ulaşmaya ne ölçüde elverişli olduğu ortaya konulur. Yöntem bölümünün; araştırma tasarımı, bağımlı ve bağımsız değişkenler, istatistiksel yöntemler vb. unsurları içermesi gerekir. Proje konusu ile ilgili ön çalışma yapılmış olması halinde bilgi verilmesi beklenir. Yönteme ilişkin akış şeması araştırmanın tasarımı göz önünde bulundurularak sunulabilir.

\begin{framed}
\lipsum[1-5]
\end{framed}
%%%%%%%%%%%%%%%%%%%%%%%%%%%%%%%%%%%%%%%%%%%%%%%%%%%%%%%%%%%%%%%%%%%%%%%%

%%%%%%%%%%%%%%%%%%%%%%%%%%%%%%%%%%%%%%%%%%%%%%%%%%%%%%%%%%%%%%%%%%%%%%%%
\begin{landscape}
\thispagestyle{mylandscape}

\section{PROJE YÖNETİMİ} 

\subsection{Yönetim Düzeni: İş-Zaman Çizelgesi ve İş Paketleri}

\subsubsection{İş-Zaman Çizelgesi}
%Projenin iş paketlerinin (İP) hangi zaman aralıklarında gerçekleştirileceği “İş-Zaman Çizelgesi”nde sunulur. Literatür taraması, gelişme ve sonuç raporu hazırlama aşamaları, proje sonuçlarının paylaşımı, makale yazımı ve malzeme alımı iş paketi olarak \underline{sunulmamalıdır}.


\begin{center}
\textbf{İŞ-ZAMAN ÇİZELGESİ (*)}\\
\vspace{0.5cm}
\setlength{\tabcolsep}{0.1pt}
\begin{tabular}{|p{0.5cm}|p{2.6cm}|p{2.6cm}|p{3.5cm}|*{36}{M{0.4cm}|}}
\hline
\rowcolor{Gray}
& & & \multicolumn{1}{Q{3.6cm}}{\bf Projenin} &
\multicolumn{36}{|c|}{}
\\ 
\rowcolor{Gray}
\multicolumn{1}{|M{0.5cm}}{\bf İP} &
\multicolumn{1}{|Q{2.6cm}}{\bf İş Paketi} &
\multicolumn{1}{|Q{2.6cm}}{\bf Başarısındaki} &
\multicolumn{1}{|M{3.6cm}}{\bf Kim(ler) Tarafından} &
\multicolumn{36}{|c|}{\textbf{AYLAR}}
\\ 
\hhline{>{\arrayrulecolor{Gray}}*{4}{-}|>{\arrayrulecolor{black}}*{36}{-}}
\rowcolor{Gray}
\multicolumn{1}{|M{0.5cm}}{\bf No} &
\multicolumn{1}{|Q{2.6cm}}{\bf Adı} &
\multicolumn{1}{|Q{2.6cm}}{\bf Önemli (\%) (**)} &
\multicolumn{1}{|Q{3.6cm}|}{\bf Gerçekleştirileceği (***)} &
1 & 2 & 3 & 4 & 5 & 6 & 7 & 8 & 9 & 10 & 11 & 12 & 13 & 14 & 15 & 16 & 17 & 18 & 19 & 20 & 21 & 22 & 23 & 24 & 25 & 26 & 27 & 28 & 29 & 30 & 31 & 32 & 33 & 34 & 35 & 36 \\ \hline
\end{tabular}
\flushleft{\footnotesize 
(*) Çizelgedeki satırlar gerektiği kadar genişletilebilir ve çoğaltılabilir.\newline
(**) Bu bölüme 0-100 arası sayısal değerler verilerek sütun toplamının 100 olması gerekmektedir.\newline
(***) İP’de görev alacak kişilerin isimleri ve görevleri (araştırmacı, danışman, bursiyer ve yardımcı personel) yazılır. Bu aşamada bursiyer(ler)in isimlerinin belirtilmesi zorunlu değildir. Proje ekibinde araştırmacı ve danışman olarak görev alacak kişiler çevrim içi başvuru sırasında başvuru ekranındaki "Proje Personeli" adımından Proje Başvuru Sistemi'ne (PBS) eklenmeli, yardımcı personel ve bursiyer olacak kişiler için ise başvuru formunun "Ek-2 Bütçe ve Gerekçesi" dokümanında bilgi verilmelidir.
}
\end{center}

\end{landscape}
\clearpage
\pagestyle{fancy}
%%%%%%%%%%%%%%%%%%%%%%%%%%%%%%%%%%%%%%%%%%%%%%%%%%%%%%%%%%%%%%%%%%%%%%%%%%%%%%%%%%%%%%%%%%%%%%
\subsubsection{İş Paketleri}

%Aşağıdaki İş Paketi Tablosu her bir İP için hazırlanır. İP’nin başarılı bir şekilde tamamlanma durumunun izlenebilmesi için her bir İP’nin hedefi, kim(ler) tarafından gerçekleştirileceği, başarı ölçütü, ara çıktısı/çıktıları ve risk yönetimi sunulur.
\begin{center}
\begin{tabular}{|p{0.24\textwidth}|p{0.24\textwidth}|p{0.24\textwidth}|p{0.24\textwidth}|}
\hline
\rowcolor{Gray}
\multicolumn{4}{|c|}{\textbf{İŞ PAKETİ TABLOSU (*)}} \\ \hline
\textbf{İP No:} 1& \multicolumn{3}{l|}{\textbf{İP Adı:} }
\\ \hline
\multicolumn{4}{|l|}{\textbf{İP Hedefi:} 
...
} \\ \hline
\multicolumn{2}{|p{0.48\textwidth}}{\textbf{İP Kapsamında Yapılacak İşler/Görevler:} 
\begin{itemize}
\item asd
\item asd
\end{itemize}
}
&
\multicolumn{2}{|p{0.48\textwidth}|}{\textbf{İP'yi Gerçekleştirecek Kişi(ler) ve İP'ye Katkıları (**)} 
\begin{itemize}
\item asd
\item asd
\end{itemize}
}
\\ \hline
\multicolumn{4}{|p{0.96\textwidth}|}{\textbf{Başarı Ölçütü:}
%İlgili iş paketinin hangi kriterleri sağladığında başarılı sayılacağı ölçülebilir ve izlenebilir şekilde nitel ve/veya nicel olarak belirtilir. 
\begin{itemize}
\item asd
\item asd
\end{itemize}
}
\\ \hline
\multicolumn{4}{|p{0.96\textwidth}|}{\textbf{Ara Çıktılar:}
%İP için öngörülen ve başarı ölçütünün gerçekleşeceğini somut olarak gösteren (teknik rapor, liste, diyagram, analiz/ölçüm sonucu, algoritma, yazılım, anket formu, verim, ham veri vb.) ara çıktılara ilişkin bilgi verilir.
\begin{itemize}
\item asd
\item asd
\end{itemize}
}
\\ \hline
\multicolumn{4}{|p{0.96\textwidth}|}{\textbf{Risk Yönetimi (***):} ...
}
%İlgili iş paketi kapsamında projenin başarısını olumsuz yönde etkileyebilecek riskler ve bu risklerle karşılaşıldığında projenin başarıyla yürütülmesini sağlamak için alınacak tedbirler (B Planı) sunulur. B planının uygulanması projenin temel hedeflerinden ve özgün değerinden sapmaya yol açmamalıdır. B planına geçilmesi durumunda yöntem değişikliğine gidiliyor ise bu durum detaylandırılmalıdır. Her bir iş paketi için risk öngörülmesi zorunlu değildir.
\\ \hline
\multicolumn{2}{|M{0.48\textwidth}}{\textbf{Risklerin Tanımı}}
&
\multicolumn{2}{|M{0.48\textwidth}|}{\textbf{Alınacak Tedbir(ler) (B planı)}}
\\ \hline
\multicolumn{2}{|p{0.48\textwidth}}{\textbullet~ asd}
&
\multicolumn{2}{|p{0.48\textwidth}|}{\textbullet~ asd}
\\ \hline
\multicolumn{2}{|p{0.48\textwidth}}{\textbullet~ asd}
&
\multicolumn{2}{|p{0.48\textwidth}|}{\textbullet~ asd}
\\ \hline
\end{tabular}
\end{center}
\flushleft{\footnotesize 
(*) İP sayısına göre tablo, risk ve alınacak tedbir (B planı) sayısına göre ilgili satırlar çoğaltılabilir. \newline
(**) İşler/Görevler’de görev alacak kişi(ler)in isimleri ve görevleri (araştırmacı, danışman, bursiyer ve yardımcı personel) yazılır. Bu aşamada, bursiyer(ler)in isimlerinin belirtilmesi zorunlu değildir. Proje yürütücüsü tüm iş paketlerinden sorumludur. Projede görevlendirilecek kişi(ler)in ilgili İP’ye sağlayacağı katkı, uzmanlık alan(lar)ına göre belirtilir.\newline
(***) Risk ve alınacak tedbir (B planı) sayısına göre ilgili satırlar çoğaltılabilir.
}
%%%%%%%%%%%%%%%%%%%%%%%%%%%%%%%%%%%

%%%%%%%%%%%%%%%%%%%%%%%%%%%%%%%%%%%%
\subsection{Araştırma Olanakları}

%Projenin yürütüleceği ve/veya proje ekibinin görev aldığı kurum/kuruluşlarda yer alan altyapı/ekipman, kullanım amacı ile birlikte listelenir.

\begin{center}
\textbf{ARAŞTIRMA OLANAKLARI TABLOSU (*)}
\begin{tabular}{|p{0.31\textwidth}|p{0.31\textwidth}|p{0.31\textwidth}|}
\hline
\rowcolor{Gray}
\multicolumn{1}{|Q{0.31\textwidth}}{\textbf{Altyapı/Ekipman Türü, Modeli}\newline (Laboratuvar, Makine-Teçhizat, vb.)} &
\multicolumn{1}{|Q{0.31\textwidth}}{\textbf{Yer Aldığı Yürütücü/Katılımcı\newline Kurum/Kuruluş}} &
\multicolumn{1}{|Q{0.31\textwidth}|}{\textbf{Projede Kullanım Amacı}}
\\ \hline
Asd & Asd  & Asd 
\\ \hline
\end{tabular}
\flushleft{\footnotesize (*) Tablodaki satırlar gerektiği kadar çoğaltılabilir.}
\end{center}

\section{YAYGIN ETKİ}

%Proje başarıyla gerçekleştirildiği takdirde projeden elde edilmesi öngörülen çıktılar ve projeden oluşması öngörülen etkiler ile bu çıktıların paylaşımı ve yayılımına yönelik etkinlikler kısa ve net olarak ilgili bölümde belirtilir. 

\subsection{Öngörülen Çıktılar}

%Projeden elde edilmesi öngörülen çıktılar amaçlarına göre belirlenen kategorilere ayrılarak belirtilir, ölçülebilir ve gerçekçi hedeflere dayandırılır. Bu çıktıları kullanacak kurum/kuruluş(lar)a (var ise) ilişkin bilgi verilmesi beklenir. Her bir çıktının elde edilmesinin öngörüldüğü zaman aralığı belirtilir.

\begin{center}
\begin{tabular}{|m{0.31\textwidth}|m{0.31\textwidth}|m{0.31\textwidth}|}
\hline
\multicolumn{1}{|Q{0.31\textwidth}}{\textbf{Çıktı Türü}} &
\multicolumn{1}{|M{0.31\textwidth}}{\textbf{Öngörülen Çıktı(lar)}} &
\multicolumn{1}{|M{0.31\textwidth}|}{\textbf{Öngörülen Zaman Aralığı (*)}}
\\ \hline
\multicolumn{1}{|P{0.31\textwidth}|}{\textbf{Bilimsel/Akademik Çıktılar}\newline
(Ulusal/Uluslararası Makale, Kitap, Kitap Bölümü, Bildiri vb.): } & &
\\ \hline
\multicolumn{1}{|P{0.31\textwidth}|}{\textbf{Ekonomik/Ticari/Sosyal Çıktılar}\newline (Prototip, Ürün, Patent, Faydalı Model, Üretim İzni, Tescil, Görsel/İşitsel Arşiv, Envanter/Veri Tabanı/Belgeleme Üretimi, Spin-off/Start- up Şirket vb.): } 
& &
\\ \hline
\multicolumn{1}{|P{0.31\textwidth}|}{\textbf{Araştırmacı Yetiştirilmesine ve Yeni Proje(ler) Oluşturulmasına Yönelik Çıktılar}\newline
(Yüksek Lisans/Doktora/Tıpta Uzmanlık/Sanatta Yeterlik Tezleri ve Ulusal/Uluslararası Yeni Proje vb.): } 
& &
\\ \hline
\end{tabular}
\flushleft{\footnotesize (*) (*) 0-12 ay, 12-18 ay, proje sonrası vb. şeklinde belirtilir.}
\end{center}

%%%%%%%%%%%%%%%%%%%%%%%%%%%%%%%%%%%%%%%%%%%%%%%%%%%%%%%%%%%%%%%%%%%%%%%%%%%%%%%%%%%%%%%%%%%%%%%
\subsection{Öngörülen Etkiler}
\setlength{\parindent}{15pt}            
%Proje başarıyla gerçekleştirildiği takdirde öngörülen uygulama alanları ve projenin sosyo-ekonomik/kültürel alanlarda sağlayacağı katkılara ilişkin değerlendirmelere yer verilir. 

%  * Öngörülen Uygulama Alanları: Projeden elde edilmesi planlanan araştırma çıktılarının mevcut ve/veya öngörülen potansiyel uygulama alanları belirtilir. Varsa proje sonuçlarından yararlanacak olası son kullanıcılarla (politika yapıcılar, sivil toplum/kullanıcılar, özel sektör vb.) ilişki kurulması ve bu ilişkinin açıklanması beklenir.
%  * Sosyo-ekonomik/Kültürel Katkı: Yaşam kalitesine katkı; kesintisiz ve güvenilir enerji arzı; temiz ve döngüsel ekonomi uygulamaları; sera gazı salınımının azaltılması; atık yönetiminin etkinleştirilmesi; iklim değişikliği ile uyum ve mücadeleye katkı; kaliteli ve güvenli temiz suya erişim; biyoçeşitliliğin korunması; sürdürülebilir, kaliteli ve güvenli gıdaya erişim; doğal afet yönetimi; sürdürülebilir ve akıllı ulaşım; kültür ve doğa varlıklarının korunması; dezavantajlı grupların toplumsal hayata katılımı; eğitim kalitesinin iyileştirilmesi; yaşam boyu öğrenme; sosyal politikalara katkı; sivil güvenlik vb. alanlarda değerlendirmeler yapılır.

%Öngörülen katkıların On İkinci Kalkınma Planı başta olmak üzere üst politika belgelerindeki hedefler ve politikalar ile ilişkisinin kurulması ve bu ilişkinin ilgili belgelere atıf yapılarak açıklanmasında fayda görülmektedir. 

\begin{framed}
\lipsum[1-3]
\end{framed}


\subsection{Proje Sonuçlarının Yayılımı ve Bilim İletişimi Kapsamında Gerçekleştirilecek Faaliyet Planı}

\begin{kutudortlu}
\noindent\textbf{Hedef Kitle:} %Proje sürecinde elde edilecek çıktı ve ulaşılacak sonuçlardan yararlanması öngörülen hedef kitlenin (akademisyenler, politika yapıcılar ve uygulayıcılar, özel sektör, bireyler, belirli yaş grupları vb.) kimler olduğu, ilgili hedef kitleye ulaşmak için nasıl bir yol izleneceği ve hedef kitlenin öngörülen yayılım faaliyetlerinden nasıl yararlanacağı belirtilir.

\tcbline*

\noindent\textbf{Hedefler ve Beklenen Kazanımlar:} %Gerçekleştirilecek yayılım faaliyetleri ile proje konusuna ilişkin farkındalığın, ilginin ve bu doğrultuda bilgi birikiminin artırılmasına yönelik nasıl bir hedef ortaya konulduğu açıklanır. Proje sonuçlarının hedef kitle ile paylaşılmasının neden önemli olduğu ve nasıl bir kazanım sağlanacağı açıklanır.

\tcbline*

\noindent\textbf{Kullanılacak Araçlar:} %Aktarılmak istenen içeriğin hangi kanallar/iletişim araçları (dijital platformlar, medya araçları, web sitesi, çalıştay, toplantı, podcast, infografik gibi görsel/işitsel araçlar, fuarlar, atölyeler, sergiler vb.) kullanılarak paylaşılacağı, neden bulundurularak açıklanır.

\tcbline*

\noindent\textbf{Zamanlama:} %Planlanan faaliyetlerin hangi zaman diliminde gerçekleştirileceği ve ne kadar süreceği açıklanır. 

\end{kutudortlu}

\textbf{\large BELİRTMEK İSTEDİĞİNİZ DİĞER KONULAR}\\

%Sadece proje önerisinin değerlendirilmesine katkı sağlayabilecek bilgi veya veri (grafik, tablo, vb.) eklenebilir.

\begin{framed}

\end{framed}

\noindent\textbf{\large BAŞVURU FORMU EKLERİ}\\

\noindent\textbf{EK-1: KAYNAKLAR}\newline
\noindent\textbf{EK-2: BÜTÇE VE GEREKÇESİ}\newline
\noindent\textbf{EK-3: PROJE EKİBİNİN DİĞER PROJELERİ VE GÜNCEL YAYINLARI (Proje Başvuru Sistemi (PBS)’ne girilen
bilgiler doğrultusunda Sistem tarafından otomatik olarak oluşturulmaktadır.)}




\clearpage
\setlength{\parindent}{0pt}            
\setcounter{page}{1}
\bibliographystyle{tubitak}
%\bibliography{refs}


\end{document}

