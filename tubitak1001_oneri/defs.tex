\usepackage[T1]{fontenc}
\usepackage[utf8]{inputenc}
\usepackage{uarial}
\renewcommand{\familydefault}{\sfdefault}
\usepackage[fontsize=9pt]{scrextend}
%%%%%%%%%%%%%%%%%%%%%%%%%%%%%%%%%%%%%%%%%%%%%%%%%%%%%%%%%%%%%
\usepackage{amsmath}
\usepackage{fancyhdr}
\usepackage{graphics}
\usepackage{xcolor,colortbl}
\usepackage{tabularx,multirow,hhline}
\usepackage{framed}
\usepackage{caption}
\usepackage{pdflscape}
\usepackage{makecell}
\usepackage{lipsum} 
%%%%%%%%%%%%%%%%%%%%%%%%%%%%%%%%%%%%%%%%%%%%%%%%%%%%%%%%%%%%%%%%%%%%
%%%%%%%%%%%
\usepackage{titlesec}
\titleformat{\section}
  {\normalfont\fontsize{9}{15}\bfseries}{\thesection.}{1em}{}
\titleformat{\subsection}
  {\normalfont\fontsize{9}{12}\bfseries}{\thesubsection.}{1em}{}
\titleformat{\subsubsection}
  {\normalfont\fontsize{9}{12}\bfseries}{\thesubsubsection.}{1em}{}



\usepackage{caption}
\renewcommand{\figurename}{Şekil}
\renewcommand{\tablename}{Tablo}

%\renewcommand{\thefigure}{\arabic{section}.\arabic{subsection}.\arabic{subsubsection}.\arabic{figure}}
%\DeclareCaptionFormat{llap}{\llap{#1#2}#3\par}
%\captionsetup{format=llap,labelsep=period,singlelinecheck=no}
\captionsetup [figure]{labelfont=bf,font=singlespacing,font=small,
position=bottom,singlelinecheck=true,aboveskip=12pt,belowskip=12pt}
\captionsetup [table]{format=hang,labelfont=bf,font=singlespacing,font=small,
position=above,singlelinecheck=true,aboveskip=12pt,belowskip=12pt}
%%%%%%%%%%%%%%%%%%%%%%%%%%%%%%%%%%%%%%%%%%%%%%%%%%%%%%%%%%%%%%%%%%%%%%%%%%%%%%%%%%

\usepackage[bookmarks, colorlinks, breaklinks, pdftitle={Tubitak 1001},pdfauthor={ },
            citecolor=blue]{hyperref}
\usepackage[authoryear,sort&compress,round]{natbib}
\usepackage{url,doi}
\renewcommand{\bibsection}{\textbf{EK-1: KAYNAKLAR}}
%%%%%%%%%%%%%%%%%%%%%%%%%%%%%%%%%%%%%%%%%
\usepackage[skip=0.6\baselineskip]{parskip}
\usepackage[top=3.0cm,headheight=1.5cm,headsep=0.5cm,
            left=2cm,textwidth=17.2cm,textheight=24cm]{geometry}
\usepackage{indentfirst}
\setlength{\parindent}{15pt}            
%%%%%%%%%%%%%%%%%%%%%%%%%%%%%%%%%
\pagestyle{fancy}
\renewcommand{\headrulewidth}{0pt}
\fancyhf{}
\cfoot{\vspace{2pt}\thepage}
\chead{\includegraphics{./tubitak.jpg}}
\lfoot{\small 1001--BF--01  Güncelleme Tarihi: 28/01/2025} %font 8
%%%%%%%%%%%%%%%%%%%%%%%%%%%%%%%%%%%%%%%%%%%%%%%%%%%%%%%%%%%%%%%%%
\fancypagestyle{mylandscape}{
\fancyhf{} %Clears the header/footer
\fancyhead[C]{
\makebox[\textwidth][l]{
\rlap{
\smash{
\raisebox{-4.87in}{
\rotatebox{90}{\includegraphics{./tubitak.jpg}}
}}}
}}
\fancyfoot[L]{% Footer
\smash{
\hspace{17.5cm}
\rotatebox{90}{\small 1001--BF--01  Güncelleme Tarihi: 28/01/2025}}}
\fancyfoot[C]{
\makebox[\textwidth][r]{% Right
%  \rlap{
%  \hspace{.75cm}% Push out of margin by \footskip
%  \smash{% Remove vertical height
%  \raisebox{4.87in}{% Raise vertically
%  \rotatebox{90}{\small 3501BF-01 Güncelleme Tarihi: 30/09/2019}}}}
%
   \rlap{
   \hspace{.75cm}% Push out of margin by \footskip
    \smash{% Remove vertical height
    \raisebox{4.87in}{% Raise vertically
    \rotatebox{90}{\thepage}}}}
    }}% Rotate counter-clockwise
\renewcommand{\headrulewidth}{0pt}% No header rule
\renewcommand{\footrulewidth}{0pt}% No footer rule
}
%%%%%%%%%%%%%%%%%%%%%%%%%%%%%%%%%%%%%%%%%%

\definecolor{Gray}{gray}{0.85}
\newcolumntype{M}[1]{>{\centering\arraybackslash}m{#1}}
\newcolumntype{Q}[1]{>{\columncolor{Gray}\centering\arraybackslash}m{#1}}
\newcolumntype{P}[1]{>{\columncolor{Gray}\arraybackslash}p{#1}}
%%%%%%%%%%%%%%%%%%%%%%%%%%%%%%%%%%%%%%%%%



\usepackage{tcolorbox}
\tcbuselibrary{skins,breakable}

\newtcolorbox{ozet}[1][]{%
colback=white,colframe=black,
width=\textwidth,
sharp corners,
breakable,
bottomrule=0.5pt, 
toprule=0.5pt,  % thicknesses of various lines
leftrule=0.5pt, 
rightrule=0.5pt,
left=6pt,
right=6pt,  
parbox=false,
segmentation style={solid},  
#1
}

\newtcolorbox{summary}[2][]{%
colbacktitle=white,
coltitle=black,
colback=white,
colframe=black,
width=\textwidth,
sharp corners,
breakable,
titlerule=0.5pt,
bottomrule=0.5pt, 
toprule=0.5pt,  % thicknesses of various lines
leftrule=0.5pt, 
rightrule=0.5pt,
left=6pt,
right=6pt,  
parbox=false,
segmentation style={solid},  
title={\textbf{Title:} #2}
#1
}

\newtcolorbox{kutudortlu}[1][]{%
colback=white,colframe=black,
width=\textwidth,
sharp corners,
breakable,
bottomrule=0.5pt, 
toprule=0.5pt,  % thicknesses of various lines
leftrule=0.5pt, 
rightrule=0.5pt,
left=6pt,
right=6pt,  
parbox=false,
segmentation style={solid},
lowerbox=invisible,
#1
}

%%%%%%%%%%%%%%%%%%%%%%%%%%%%%%%
\usepackage{pgfmath}  
\usepackage{xstring}  
\usepackage{refcount} 

\newcommand{\refte}[1]{%
  \IfSubStr{\getrefnumber{#1}}{.}{%
    \StrBehind{\getrefnumber{#1}}{.}[\eqnum]}%     % Extract part after dot
  { \def\eqnum{\getrefnumber{#1}} }%   
  \ifx\eqnum\empty%
   \autoref{#1}'te
  \else%
  \pgfmathparse{int(mod(\eqnum , 10))}% Extract last digit
  \let\lastdigit\pgfmathresult%
  \pgfmathparse{int(mod(\eqnum , 100))} % Extract last digit
  \let\lasttwodigit\pgfmathresult%
  \autoref{#1}'%
  \ifnum\lastdigit=0%
      \ifnum\lasttwodigit=0 de%
      \else\ifnum\lasttwodigit=10 da%
      \else\ifnum\lasttwodigit=20 de%
      \else\ifnum\lasttwodigit=30 da%
      \else\ifnum\lasttwodigit=40 ta%
      \else\ifnum\lasttwodigit=50 de%
      \else\ifnum\lasttwodigit=60 ta%
      \else\ifnum\lasttwodigit=70 te%
      \else\ifnum\lasttwodigit=80 de%
      \else\ifnum\lasttwodigit=90 da%
      \fi\fi\fi\fi\fi\fi\fi\fi\fi\fi
  \else\ifnum\lastdigit=1 de%
  \else\ifnum\lastdigit=2 de%
  \else\ifnum\lastdigit=3 te%
  \else\ifnum\lastdigit=4 te%
  \else\ifnum\lastdigit=5 te%
  \else\ifnum\lastdigit=6 da%
  \else\ifnum\lastdigit=7 de%
  \else\ifnum\lastdigit=8 de%
  \else\ifnum\lastdigit=9 da%
  \fi\fi\fi\fi\fi\fi\fi\fi\fi\fi
\fi
}

\newcommand{\refin}[1]{%
  \IfSubStr{\getrefnumber{#1}}{.}{%
    \StrBehind{\getrefnumber{#1}}{.}[\eqnum]}%     % Extract part after dot
  { \def\eqnum{\getrefnumber{#1}} }%   
  \ifx\eqnum\empty
   \autoref{#1}'in
  \else
  \pgfmathparse{int(mod(\eqnum , 10))}% % Extract last digit
  \let\lastdigit\pgfmathresult%
  \autoref{#1}'%
  \ifnum\lastdigit=0%
      \pgfmathparse{int(mod(\eqnum , 100))}% % Extract last digit
      \let\lasttwodigits\pgfmathresult%
      \ifnum\lasttwodigits=0 ün%
      \else\ifnum\lasttwodigits=10 un% 
      \else\ifnum\lasttwodigits=20 nin%
      \else\ifnum\lasttwodigits=30 un%
      \else\ifnum\lasttwodigits=40 ın%
      \else\ifnum\lasttwodigits=50 nin%
      \else\ifnum\lasttwodigits=60 ın%
      \else\ifnum\lasttwodigits=70 in%
      \else\ifnum\lasttwodigits=80 nin%
      \else\ifnum\lasttwodigits=90 nın%
      \fi\fi\fi\fi\fi\fi\fi\fi\fi\fi 
  \else\ifnum\lastdigit=1 in%
  \else\ifnum\lastdigit=2 nin%
  \else\ifnum\lastdigit=3 ün%
  \else\ifnum\lastdigit=4 ün%
  \else\ifnum\lastdigit=5 in%
  \else\ifnum\lastdigit=6 nın%
  \else\ifnum\lastdigit=7 nin%
  \else\ifnum\lastdigit=8 in%
  \else\ifnum\lastdigit=9 un%
  \fi\fi\fi\fi\fi\fi\fi\fi\fi\fi
\fi
}


\newcommand{\refi}[1]{%
  \IfSubStr{\getrefnumber{#1}}{.}{%
    \StrBehind{\getrefnumber{#1}}{.}[\eqnum]}%     % Extract part after dot
  { \def\eqnum{\getrefnumber{#1}} }%   
  \ifx\eqnum\empty
   \autoref{#1}'i
  \else
  \pgfmathparse{int(mod(\eqnum , 10))}% % Extract last digit
  \let\lastdigit\pgfmathresult%
  \autoref{#1}'%
  \ifnum\lastdigit=0%
      \pgfmathparse{int(mod(\eqnum , 100))}% % Extract last digit
      \let\lasttwodigits\pgfmathresult%
      \ifnum\lasttwodigits=0 ü%
      \else\ifnum\lasttwodigits=10 u% 
      \else\ifnum\lasttwodigits=20 yi%
      \else\ifnum\lasttwodigits=30 u%
      \else\ifnum\lasttwodigits=40 ı%
      \else\ifnum\lasttwodigits=50 yi%
      \else\ifnum\lasttwodigits=60 ı%
      \else\ifnum\lasttwodigits=70 i%
      \else\ifnum\lasttwodigits=80 i%
      \else\ifnum\lasttwodigits=90 ı%
      \fi\fi\fi\fi\fi\fi\fi\fi\fi\fi % 
  \else\ifnum\lastdigit=1 i%
  \else\ifnum\lastdigit=2 yi%
  \else\ifnum\lastdigit=3 ü%
  \else\ifnum\lastdigit=4 ü%
  \else\ifnum\lastdigit=5 i%
  \else\ifnum\lastdigit=6 yı%
  \else\ifnum\lastdigit=7 yi%
  \else\ifnum\lastdigit=8 i%
  \else\ifnum\lastdigit=9 u%
  \fi\fi\fi\fi\fi\fi\fi\fi\fi\fi
\fi
}


\newcommand{\refe}[1]{%
  \IfSubStr{\getrefnumber{#1}}{.}{%
    \StrBehind{\getrefnumber{#1}}{.}[\eqnum]}%     % Extract part after dot
  { \def\eqnum{\getrefnumber{#1}} }%   
  \ifx\eqnum\empty%
   \autoref{#1}'e%
  \else%
  \pgfmathparse{int(mod(\eqnum , 10))}% % Extract last digit
  \let\lastdigit\pgfmathresult%
  \autoref{#1}'%
  \ifnum\lastdigit=0%
      \pgfmathparse{int(mod(\eqnum , 100))}% % Extract last digit
      \let\lasttwodigits\pgfmathresult%
      \ifnum\lasttwodigits=0 e%
      \else\ifnum\lasttwodigits=10 a% 
      \else\ifnum\lasttwodigits=20 ye%
      \else\ifnum\lasttwodigits=30 a%
      \else\ifnum\lasttwodigits=40 a%
      \else\ifnum\lasttwodigits=50 ye%
      \else\ifnum\lasttwodigits=60 a%
      \else\ifnum\lasttwodigits=70 e%
      \else\ifnum\lasttwodigits=80 e%
      \else\ifnum\lasttwodigits=90 a%
      \fi\fi\fi\fi\fi\fi\fi\fi\fi\fi
  \else\ifnum\lastdigit=1 e%
  \else\ifnum\lastdigit=2 ye%
  \else\ifnum\lastdigit=3 e%
  \else\ifnum\lastdigit=4 e%
  \else\ifnum\lastdigit=5 e%
  \else\ifnum\lastdigit=6 ya%
  \else\ifnum\lastdigit=7 ye%
  \else\ifnum\lastdigit=8 e%
  \else\ifnum\lastdigit=9 a%
  \fi\fi\fi\fi\fi\fi\fi\fi\fi\fi
  \fi
}

\newcommand{\reften}[1]{%
  \IfSubStr{\getrefnumber{#1}}{.}{%
    \StrBehind{\getrefnumber{#1}}{.}[\eqnum]}%     % Extract part after dot
  { \def\eqnum{\getrefnumber{#1}} }%   
  \ifx\eqnum\empty%
   \autoref{#1}'ten%
  \else%
  \pgfmathparse{int(mod(\eqnum , 10))}% % Extract last digit
  \let\lastdigit\pgfmathresult%
  \autoref{#1}'%
  \ifnum\lastdigit=0% 
      \pgfmathparse{int(mod(\eqnum , 100))}% % Extract last digit
      \let\lasttwodigits\pgfmathresult%
      \ifnum\lasttwodigits=0 dan%
      \else\ifnum\lasttwodigits=10 dan% 
      \else\ifnum\lasttwodigits=20 den%
      \else\ifnum\lasttwodigits=30 dan%
      \else\ifnum\lasttwodigits=40 tan%
      \else\ifnum\lasttwodigits=50 den%
      \else\ifnum\lasttwodigits=60 tan%
      \else\ifnum\lasttwodigits=70 ten%
      \else\ifnum\lasttwodigits=80 den%%
      \else\ifnum\lasttwodigits=90 dan%
      \fi\fi\fi\fi\fi\fi\fi\fi\fi\fi%
  \else\ifnum\lastdigit=1 den%
  \else\ifnum\lastdigit=2 den%
  \else\ifnum\lastdigit=3 ten%
  \else\ifnum\lastdigit=4 ten%
  \else\ifnum\lastdigit=5 ten%
  \else\ifnum\lastdigit=6 dan%
  \else\ifnum\lastdigit=7 den%
  \else\ifnum\lastdigit=8 den%
  \else\ifnum\lastdigit=9 dan%
  \fi\fi\fi\fi\fi\fi\fi\fi\fi\fi%
\fi%
}